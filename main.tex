%
% File ACL2016.tex
%

\documentclass[11pt]{article}
\usepackage{acl2016}
\usepackage{times}
\usepackage{latexsym}
\usepackage{url}
\usepackage{booktabs}
\usepackage{graphicx}
\usepackage{color}
\usepackage{amsmath}
\aclfinalcopy 

\usepackage[authoryear]{natbib}
\usepackage{url}

\title{YOUR PROJECT TITLE}
\author{ADD YOUR NAME AND STUDENT NUMBER\\
Length: 3-4 pages}
\date{}

\begin{document}
\maketitle

%%% YOUR PART HERE
\begin{abstract}
\textcolor{red}{Your abstract here (max 10-15 lines).}
\end{abstract}

%% IMPORTANT: KEEP ALL SECTIONS (headers)
%% remove the 'red' text parts

\section{Introduction}

\textcolor{red}{
\begin{itemize}
\item This template is based on the template of previous iterations of the course.
\item Introduce your research (context)
\item Problem statement and research question, including concrete hypothesis
\end{itemize}
} 

\section{Related Work}

\textcolor{red}{Use peer-reviewed sources relevant to your own project, discuss them in relation to your own work and cite properly. If you find additional papers, discuss them (optional).}

\textcolor{red}{You can cite a paper by using BibTeX like this~\citep{Culotta:ea:2014}, or the following way if the paper is part of the sentence like~\cite{Culotta:ea:2014}, where the BiBTeX entry is in the \texttt{mybib.bib} file. The proper use of \LaTeX{} and citation using BiBteX is part of this assignment.} 

\section{Data}

\textcolor{red}{Describe the data you will use in your research. Describe the data collection, the way you obtained the part of the data you would work with (precisely describe how you will obtain the data for your independent and dependent variables). You can provide a table with examples to give the reader an idea of the dataset you are using. For example, ``We will construct ..."\footnote{\textcolor{red}{[Note: use `we' rather than `I', even though you are the only author (this work is individually done!) - this is common practice in many fields]}}}

\textcolor{red}{Describe how you would obtain data for your variables. If it relies on an automatic way to inferring the variable (e.g., for gender) describe precisely how you would do it. Mention also possible limitations of the way you would obtain the data.}

\paragraph{Pre-processing} \textcolor{red}{Describe how you would pre-process the data. Describe all design choices (e.g., would you tokenize the data? how would you handle retweets?).}

Table~\ref{tbl:stats} provides a summary of the data that will be used in this study. 
\begin{table}[hbtp]\centering
\begin{tabular}{|cc|}
\hline
A & B \\
\hline
A & B \\
A & B \\
\hline
\end{tabular}
\caption{Overview of the data set. Provide an appropriate caption.}
\label{tbl:stats}
\end{table}


\section{Predicted Results}

\textcolor{red}{In this section you will describe your expected results based on the literature you found.}

Table~\ref{tbl:results} summarizes...


\begin{table}[hbtp]\centering
\begin{tabular}{|ccc|}
\hline
A & B & C\\
\hline
A & B & C\\
A & B & C\\
\hline
\end{tabular}
\caption{Add a caption}
\label{tbl:results}
\end{table}


\paragraph{Discussion} 

\textcolor{red}{This is the part in which you discuss what it would mean to find these results and their implications.}

\section{Conclusion}

\textcolor{red}{Wrap up your report by giving answers to your research question in the beginning. For example, ``This study aimed at ..." You might mention limitations of the current study, or directions for further research here.}

%%END YOUR PART

\section*{\textcolor{red}{Check list}}
\textcolor{red}{(in your final paper remove this section)}
\textcolor{red}{
\begin{itemize}
\item Make sure your paper is max 4 pages long (excluding references and appendices)
\item check that your paper contains an abstract
\item check that your paper states the research hypothesis
clearly, and defines independent and dependent variables
\item make sure your paper discusses at least 3 papers that you found relevant to your own work
\item check that your paper describes precisely how you would create
the data set that you base your research on, including all steps in preprocessing the data
\item make sure your paper links to the github repository
\item make sure your paper contains a predicted results section, which shows the predicted results of your study and discusses them (table + text).
\item finally, make sure you have all parts of the report: abstract, introduction, related work, data, predicted results, conclusion
\item check proper use of \LaTeX{} and BibTex
\end{itemize}
}

\bibliographystyle{chicago}
\bibliography{mybib.bib}

\end{document}



